% 1 page maximum
% B. Specific Aims. Develop a proposal with at least one original aim (this aim may not
% have previously appeared in any of the student’s PI’s proposals or been discussed
% previously in some depth with the PI). 

% In the Specific Aims section, one should state concisely the goals of the proposed research
% and summarize the expected outcome(s), including the impact that the results of the
% proposed research will exert on the research field(s) involved. 

%List succinctly the specific
% objectives of the research proposed, e.g., to test a stated hypothesis, create a novel design,
% solve a specific problem, challenge an existing paradigm or clinical practice, address a
% critical barrier to progress in the field, or develop new technology.
\section{Specific aims}

\paragraph{Aim \Romannum{1}: build a machine learned predictor for tuned docking parameters} 
% The goal of this aim will be to build the architecture to assist virtual screening, i.e. the new `scoring function' will output more precise drug lead prediction than using a docking software without this architecture.
In this aim, docking parameters will be generated and trained with SVM. 
The generation of docking parameters rely on the CASF (Comparative Assessment of Scoring Functions) protocol that is newly published for the evaluation of scoring functions.
A mini-batch gradient descent algorithm will be used to update the docking parameters until it passes a user-define threshold.
At the end of this step, docking parameters specific to individual  ligand-receptor complex and the complex's molecular descriptor will be used to train a SVM so that, the SVM predictor can generate docking parameters based on an input molecular descriptor.

\paragraph{Aim 2: Active selection of ligand to-be-docked in virtual screening}
Instead of randomly going through the ligands to be tested in virtual screening,
this proposals develops an active selection that involves SVM, in which all of the ligands in ligand library to be docked will be projected onto a feature space, with a hyperplane separating the so-far known active ligands and non-active ligands.
As screening processes more ligand will be labeled and the hyperplane will be dynamically updated. 
The key is to always use the ligand that's furtherest away from the hyperplane in the active ligands side.


\paragraph{Aim 3: applicate the combined virtual docking architecture for Alzheimer’s disease related receptors} 
This aim applicated the previously built architecture in the realistic scenario of searching inhibitor (binder) for proteins relevant to Alzheimer’s disease.
% Simplify the architecture from a regression model to a classification problem and applicate it into a multi-targeted directed ligands virtual screening case.
