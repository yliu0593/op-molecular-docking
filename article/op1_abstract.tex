%one paragraph (15-25 lines) separate page
% A. Project Summary. It is meant to serve as a succinct and accurate description of the
% proposed work when separated from the application. State the application's broad, longterm objectives and specific aims, making reference to the health relatedness of the
% project. Describe concisely the research design and methods for achieving the stated
% goals. This section should be informative to other persons working in the same or related
% fields and insofar as possible understandable to a scientifically or technically literate
% reader.
\section{Project Summary}

This document proposes a machine learning architecture for structure based virtual screening in drug discovery. 
In drug discovery, virtual screening is categorized into structure based and ligand based depending on the availability of structural data of the receptor.
This proposal focuses on structure based virtual screening,
and proposes a two step improvement that targets the docking step and the screening step in virtual screening 
as aim \Romannum{1} and aim \Romannum{2} respectively. 
Both of the improvement utilize a machine learning algorithm, specifically, support vector machine (SVM), 
which is considered to be a robust and well-performing algorithm 
for small datasets.
Aim \Romannum{3} suggests the application of previously constructed architecture as aim \Romannum{1} and aim \Romannum{2} combined for the screening of Alzheimer's disease related targets. 


% A frustration situation in todays \textit{in silico} screening 
% is that
% drug leads that are predicted to be active binder of the target protein 
% fail to bind in real experimental conditions.
% Generally it is summarized this is due to the limitations of the docking software,
% more specifically, the scoring functions within the docking software.
% Scoring functions have been under active development towards better estimation of binding affinity,
% and much efforts have been put into building more realistic model in molecular dynamics level, 
% or incorporating machine learning algorithms training towards better predicting scoring functions.
% However, not much efforts are put into a complex-specific configuration per docking event.
% This article uses the CASF scoring functions evaluation protocol 
% to optimize the complex specific docking configuration
% with simple machine learning design.
% Despite intensive research over the years, classical scoring functions have reached a plateauin their predictive performance.
%  These assume a predetermined additive functional form for some sophisticatednumerical features, and use standard multivariate linear regression (MLR) on experimental data to derive thecoefficients.
% This proposal takes the popular docking software AutoDock Vina and train the output of which towards better performance in the CASF-2013 benchmark datasets.