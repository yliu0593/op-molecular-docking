%% By Youlin Liu 2018-09-17 for op

\documentclass[11pt]{article}
\usepackage[scale=0.8]{geometry} %http://texdoc.net/texmf-dist/doc/latex/geometry/geometry.pdf [left=1.5cm,right=1.5cm,top=1.5cm,bottom=1.5cm]
\usepackage{chngpage}
\usepackage{indentfirst} 
\usepackage{graphicx} % Required for including images
\DeclareGraphicsExtensions{.eps,.pdf,.jpeg,.png}
\usepackage[labelfont=bf]{caption}
\usepackage{amsmath}                      % Use more than one optional parameter in a new commands
\usepackage{xargs}
\usepackage[square,numbers]{natbib} %use natbib instead of cite, \bibliographystyle{plainnat},to cite author use \citep
% \bibliographystyle{nihunsrt}
\usepackage{subcaption} % use subcaption instead of subfigure <- keeps giving me compling errors
% \usepackage[colorlinks,linkcolor=black,anchorcolor=blue,citecolor=green]{hyperref}
\usepackage[pdftex,colorlinks,linkcolor=black,citecolor=black,urlcolor=black,filecolor=black]{hyperref}
% \usepackage{courier} %for inline code snippets,for \ttfamily in listing

% \usepackage{fontspec}
% \setmainfont{Georgia} 
% \setsansfont{Trebuchet MS} 
% \setmonofont{Inconsolata}
\renewcommand{\familydefault}{\ttdefault}

%%%%%%%%%%%%%%%%%%%%%%%%%%%%%%%%%%%%%%%%%%% COLOR %%%%%%%%%%%%%%%%%%%%%%%%%%%%%%%%%%%%%%%%%%%
%%%defined colors: black, blue, brown, cyan, darkgray, gray, green, lightgray, lime, magenta, olive, orange, pink, purple, red, teal, violet, white, yellow.
\usepackage[pdftex,dvipsnames]{xcolor}
\definecolor{emph}{RGB}{255,87,51} % emphasizing text
\definecolor{mygreen}{rgb}{0,0.6,0}
\definecolor{mygray}{rgb}{0.5,0.5,0.5}
\definecolor{mymauve}{rgb}{0.58,0,0.82}
\definecolor{inp_local_comments}{RGB}{42,0.0,255} % INP local comments
\definecolor{inp_questions}{RGB}{186,19,178} % questions raised by user
\definecolor{question}{RGB}{186,19,178} % questions raised by user
\definecolor{inp_comments}{RGB}{43,140,75} % INP comments
\definecolor{inp_other}{RGB}{127,0,85} 
\definecolor{inp_key_entry}{RGB}{255,87,51} % INP important entried
% \definecolor{inp_local_comments}{RGB}{42,0.0,255} % INP local comments
% \definecolor{inp_questions}{RGB}{186,19,178} % questions raised by user
% \definecolor{question}{RGB}{186,19,178} % questions raised by user
% \definecolor{inp_comments}{RGB}{43,140,75} % INP comments
% \definecolor{inp_other}{RGB}{127,0,85} 
% \definecolor{matlab_additional_comments}{RGB}{255,87,51} % INP important entried
%%%defined colors: black, blue, brown, cyan, darkgray, gray, green, lightgray, lime, magenta, olive, orange, pink, purple, red, teal, violet, white, yellow.
%\textcolor{declared-color}{text}



%%%%%%%%%%%%%%%%%%%%%%%%%%%%%%%%%%%%%%%%%%%%% wrap figure %%%%%%%%%%%%%%%%%%%%%%%%%%%%%
\usepackage{wrapfig}%
% %usage
% \begin{wrapfigure}{r}{0.48\textwidth}%{4.5cm}%靠文字内容的左侧
% \includegraphics[width=0.445\textwidth]{figures/xds_flow.png}
% \caption{Processing file exchange in XDS}
% \label{fig:xds_flow}
% \end{wrapfigure}



%%%%%%%%%%%%%%%%%%%%%%%%%%%%%%%%%%%%% FLOW CHARTS %%%%%%%%%%%%%%%%%%%%%%%%%%%%%%%%%%%%%%%%
% \usepackage[latin1]{inputenc}
\usepackage{tikz}
\usetikzlibrary{shapes,arrows}
% Define block styles
% Define block styles
\tikzstyle{decision} = [diamond, draw, fill=blue!20, 
    text width=4.5em, text badly centered, node distance=3cm, inner sep=0pt]
\tikzstyle{block} = [rectangle, draw, fill=blue!20, 
    text width=5em, text centered, rounded corners, minimum height=4em]
    \tikzstyle{block2} = [rectangle, draw, fill=red!20, 
    text width=5em, text centered, rounded corners, minimum height=4em]
     \tikzstyle{block3} = [rectangle, draw, fill=green!20, 
    text width=5em, text centered, rounded corners, minimum height=4em]
\tikzstyle{line} = [draw, -latex']
\tikzstyle{cloud} = [draw, ellipse,fill=red!20, node distance=3cm,
    minimum height=2em]


%%%%%%%%%%%%%%%%%%%%%%%%%%%%%%%%%%%%%%%%%%%%%%%%  LISTING %%%%%%%%%%%%%%%%%%%%%%%%%%%%%%%%%%%%
\usepackage{listings}  
\lstset{               
  language=matlab,% choose the language of the code
  numbers=left,                   % where to put the line-numbers
  stepnumber=1,                   % the step between two line-numbers.        
  numbersep=5pt,                  % how far the line-numbers are from the code
  backgroundcolor=\color{white},  % choose the background color. You must add \usepackag
  basicstyle=\small\ttfamily,        % size of fonts used for the code
  breaklines=true,                 % automatic line breaking only at whitespace
  captionpos=b,                    % sets the caption-position to bottom
  commentstyle=\color{mygreen},    % comment style
  escapeinside={\%*}{*)},          % if you want to add LaTeX within your code
  keywordstyle=\color{blue},       % keyword style
  stringstyle=\color{mymauve},
  showspaces=false,               % show spaces adding particular underscores
  showstringspaces=false,         % underline spaces within strings
  showtabs=false,                 % show tabs within strings adding particular underscores
  tabsize=2,                      % sets default tabsize to 2 spaces
  captionpos=b,                   % sets the caption-position to bottom
  breaklines=true,                % sets automatic line breaking
  title=\lstname,                 % show the filename of files included with \lstinputlisting;
}



%%%%%%%%%%%%%%%%%%%%%%%%%%%%%%%%%%%%%%%%%%% TODO notes %%%%%%%%%%%%%%%%%%%%%%%%%%%%%%%%%%%%%
% oringinally from http://tex.stackexchange.com/questions/9796/how-to-add-todo-notes
\usepackage[colorinlistoftodos,prependcaption,textsize=small]{todonotes}
\newcommandx{\rep}[2][1=]{\todo[linecolor=red,backgroundcolor=red!25,bordercolor=red,#1]{#2}}
%\newcommandx{\change}[2][1=]{\todo[linecolor=blue,backgroundcolor=blue!25,bordercolor=blue,#1]{#2}}
\newcommandx{\note}[2][1=]{\todo[linecolor=OliveGreen,backgroundcolor=OliveGreen!25,bordercolor=OliveGreen,#1]{#2}}
\newcommandx{\info}[2][1=]{\todo[linecolor=Plum,backgroundcolor=Plum!25,bordercolor=Plum,#1]{#2}}
\newcommandx{\hide}[2][1=]{\todo[disable,#1]{#2}}
% \todo[inline]{The original todo note withouth changed colours.\newline Here's another line.}
% \lipsum[11]\unsure{Is this correct?}\unsure{I'm unsure about also!}
% \lipsum[11]\change{Change this!}
% \lipsum[11]\info{This can help me in chapter seven!}
% \lipsum[11]\improvement{This really needs to be improved!\newline\newline What was I thinking?!}
% \lipsum[11]
% \thiswillnotshow{This is hidden since option `disable' is chosen!}
% \improvement[inline]{The following section needs to be rewritten!}



%%%source http://tex.stackexchange.com/questions/22796/how-to-define-macro-for-colored-text
% \newcommand\red[1]{{\color{red}#1}}
% \newcommand\bft{\textbf}




%%%%%%%%%%%%%%%%%%%      numbering style
% \renewcommand\thesection{\Roman{section}}
% \renewcommand\thesubsection{\thesection.\roman{subsection}}
% \arabic (1, 2, 3, ...)
% \alph (a, b, c, ...)
% \Alph (A, B, C, ...)
% \roman (i, ii, iii, ...)
% \Roman



%%%%%%%%%%%%%%%%%%%       misc settings
% below is copied from the NIH latex template
\usepackage[scaled]{helvet} % Helvetica font for NIH
\renewcommand*\familydefault{\sfdefault} % Use the sans serif version of the font
\usepackage[T1]{fontenc}
\linespread{1.05} % A little extra line spread is better for the Palatino font
\usepackage{lipsum} % Used for inserting dummy 'Lorem ipsum' text into the template
\usepackage{amsfonts, amsmath, amsthm, amssymb} % For math fonts, symbols and environments
\hyphenation{ionto-pho-re-tic iso-tro-pic fortran} % Specifies custom hyphenation points for words or words that shouldn't be hyphenated at all
% \setlength{\parindent}{0in}
% \setlength{\parskip}{3mm}
\graphicspath{../figures/}


%%%%%%%%%%%%%%%%%%%%%%%%%make glossaryes
\usepackage[utf8]{inputenc}
\usepackage{glossaries}

\usepackage{siunitx}
% \newglossaryentry{latex}
% {
%     name=latex,
%     description={Is a mark up language specially suited 
%     for scientific documents}
% }
% \clearpage
% \printglossaries


%%%%%%%%%%%%%%%%%%%%%%%%%%%%%%%%%%%%% listing for input docking file cause i just feel like it
%%% 09-28-2018 decide not to do it: not worth it for figuring out wild cards
% just \inputlstlsting with {}
% \lstset{language=C,caption={Descriptive Caption Text},label=DescriptiveLabel}
% Define Language
% \lstdefinelanguage{plain_input}
% {
%   % list of keywords
%   morekeywords={
%   DETECTOR_DISTANCE,
%     JOB,
%     SPACE_GROUP_NUMBER,
%     NAME_TEMPLATE_OF_DATA_FRAMES, 
%     DATA_RANGE,
%     SPOT_RANGE,
%     INCLUDE_RESOLUTION_RANGE
%   },
%   sensitive=TRUE, % keywords are not case-sensitive
%   morecomment=[l]{!}, % l is for line comment
%   %morecomment=[s]{/*}{*/}, % s is for start and end delimiter
%   %morestring=[b]" % defines that strings are enclosed in double quotes
% }
% % Set Language
% \lstset{
%   language={plain_input},
%   basicstyle=\small\ttfamily, % Global Code Style
%   captionpos=b, % Position of the Caption (t for top, b for bottom)
%   extendedchars=true, % Allows 256 instead of 128 ASCII characters
%   tabsize=2, % number of spaces indented when discovering a tab 
%   columns=fixed, % make all characters equal width
%   keepspaces=true, % does not ignore spaces to fit width, convert tabs to spaces
%   showstringspaces=false, % lets spaces in strings appear as real spaces
%   breaklines=true, % wrap lines if they don't fit
%   frame=trbl, % draw a frame at the top, right, left and bottom of the listing
%   frameround=tttt, % make the frame round at all four corners
%   framesep=4pt, % quarter circle size of the round corners
%   numbers=left, % show line numbers at the left
%   numberstyle=\tiny\ttfamily, % style of the line numbers
%   commentstyle=\color{inp_comments}, % style of comments
%   keywordstyle=\color{inp_key_entry}, % style of keywords
%   stringstyle=\color{inp_other}, % style of strings
% }
% \lstset{morecomment=[l][\color{inp_local_comments}]{!--}}
% \lstset{morecomment=[l][\color{inp_questions}]{!++}}


\usepackage{romannum}





%%%%%%%%%%%%%%%%%%%%%%%%%%%%%%%%%%%%%%% CHEAT SHEET %%%%%%%%%%%%%%%%%%%%%%%%%%%%%%%%%%%%%%%%%

%%%%%%%%%%%%%%%%%%%%input aligned multiple  figures%%%%%%%%%%%%%%%%%%%%%%
	% \begin{figure}[htbp]
	% 	\begin{minipage}{0.33\textwidth}
	% 		\includegraphics[width=1.0\textwidth]{../images/ori.png}
	% 		\centerline{Original image}
	% 	\end{minipage}
	% 	\begin{minipage}{0.33\textwidth}
	% 		\includegraphics[width=1.0\textwidth]{../images/res400.png}
	% 		\centerline{Restored image with $\gamma = 4$}
	% 	\end{minipage}
	% 	\begin{minipage}{0.33\textwidth}
	% 		\begin{center}
	% 			\includegraphics[width=1.0\textwidth]{../images/dif400.png}
	% 			\centerline{Difference image with $\gamma = 4$}
	% 		\end{center}
	% 	\end{minipage}
	% 	\caption{Original image, restored image and difference image with $\gamma = 4$}
	% 	\label{3}
	% \end{figure}

%%%%%%%%%%%%%%%%%%%%%%%%%%%input 1 figure%%%%%%%%%%%%%%%%%%%%%%%%%%%
% \begin{figure}[htbp]
% \centering
% \includegraphics[width=\textwidth]{../images/AC.png}
% 		\caption{Mean AC value across blocks}
% 		\label{AC}
% 	\end{figure}